\chapter{Evaluation}

The software produced in this project has met every requirement set out in the
specification. The end result is deemed to be a significant step towards
a viable offering by \emph{Servat, Ltd} of Nottingham, who agreed to support
this project in the initial stages and have provided a detailed evaluation.

\section{Servat Evaluation}

\begin{quote}
\begin{em}

    At Servat, we are mainly an accounting systems provider, installing and
    maintaining DataFile for our clients. We also resell a remote backup
    system, however, called Vembu. This is re-branded as our own product,
    DataSquirrel. DataSquirrel backs up customer data over the internet to
    a server in the cloud.

    The software Rob has produced is of a standard we would consider marketing
    as an in-house alternative to our DataSquirrel product. There are, however,
    a number of things which prevent us from doing so.
    
    Firstly, the backup system does not support Windows operating systems,
    which is a definite must in our customer environments, which are almost
    entirely Windows-based. Also, the backups themselves are not encrypted,
    which, when dealing with sensitive accounting data, would also be required.
    Finally, the installation process is somewhat complex, and would not be
    suitable for distribution to customers as-is. This would need to be refined
    before the software could be realistically used in a commercial setting.

    Aside from these issues, the software is generally well produced and
    professional. It fulfils every requirement in the specification which was
    written at the beginning of the project, and performs well in all areas. In
    particular, the web interface is well designed and intuitive, which would
    allow less technical personnel to recover files without contacting us
    unnecessarily. All-in-all, an impressive piece of work!

\end{em}
\end{quote}

\section{Methodology}

As described in section \ref{sec:implementation-methodology}, this project was
developed using an \emph{evolutionary prototyping} methodology. Upon
reflection, this methodology allowed for freedom in the development process,
and was extremely valuable when refining the system architecture.

Despite having only limited development experience to apply, the author
considers this methodology to be very effective, enforcing good design
principles at every step of the way. Though the design was drastically
re-considered in the early stages of the project's life, the knowledge that
each prototype fed into the next iteration encourages the developer to adhere
to so-called ``good'' design principles, making use of reusable modules and
abstracted code.

Ultimately, this has led to a well-designed, extensible system, upon which the
features discussed in section \ref{sec:specification-evolution} and chapter
\ref{chap:further-work} may be developed.

\section{The Software}

The ultimate goal of this project---to produce a working backup system based on
the principles of version control and ease-of-use---was achieved. The software
delivers in all aspects which were promised, providing:

\begin{itemize}
    \item Version-based backup and recovery,
    \item A web interface for central management,
    \item Detailed reports in the form of events and catalog browsing,
    \item Instant file recovery via the web interface.
\end{itemize}

Given that the core project aims outlined in section
\ref{sec:introduction-aims} have been achieved, the project can be deemed to
have been a success.

On the road to project completion, the intermediate objectives must also be
considered. Expertise has been gained in the use and workings of the
technologies involved, multiple prototypes have been produced and an industrial
sponsor's support gained.

In summary, the backup system does everything it is supposed to do, whilst
leaving room for improvement and development upon the foundation that has been
laid.

\section{Personal Reflections}

From the outset, this project was to be an ambitious attempt to produce
a highly critical piece of software, upon which people depend when others fail.

Having previous experience developing web applications, the author's first
instincts to create an entirely web-based system turned out to be entirely
incompatible. As a result, the project was re-written multiple times, finally
achieving an acceptable result.

Certainly the most challenging aspect of this project was working with the
\emph{Twisted} framework. By all accounts, Twisted is an amazingly powerful
piece of software. Apparently, however, the developers have little time left
after creating such a marvel---as the documentation is severely lacking. After
hacking workarounds for numerous bugs, guessing at API parameters, and
unpicking countless lines of code, the author's endeavours have certainly paid
off.

Though, as always is the case with such things, a far superior and more novel
product could have been achieved given just \emph{a little more time}, the
author feels a sense of achievement in creating a system which punches well
above its weight in offering that which others do not: \emph{simple},
\emph{intuitive} and best of all \emph{free} version control for your backups!
