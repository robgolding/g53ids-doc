\chapter{Research}

\begin{bibunit}[plain]

\section{Related Works}

Related works fit into one of three main categories: Traditional backup
systems, on-line (cloud based) backup solutions, and file synchronisation
services. Ideally, this project will combine the positive elements of each of
these systems.

\subsection{Traditional Backup Systems}

\subsubsection{Amanda Network Backup}

Amanda (Advanced Maryland Automatic Network Disk Archiver) is an open-source,
multi-platform network backup solution\cite{AMANDA-about}. The software is
released under the Amanda Copyright and License (a BSD-style license) and is
supported by Zmanda\cite{AMANDA-license}.

Amanda uses native Unix tools to archive files (such as tar or dump) to disk or
tape on a central backup server. This means that an administrator can restore
the archived data, even if the software on the backup server is malfunctioning
(or indeed has failed altogether).

The client agent runs on almost all Linux or Unix-based systems, while the
server software only support Linux. Due the to fact that it only uses native
tools, no special drivers are required for interfacing with devices such as
tape libraries. Therefore, if the operating system is able to communicate with
the device (via commands such as \emph{mt}) then Amanda can use
it\cite{AMANDA-about}.

The software lends itself heavily to the use of tape devices for backing up
data, and has some advanced features that facilitate this use. As such, it does
not support a version-centric backup model or
instant-recovery\cite{AMANDA-about}.

An ``enterprise'' version is also available---supported by Zmanda---which
offers a GUI management tool and enterprise-class support (SLA).  This version
is based on an annual subscription model\cite{AMANDA-ent}.

\subsubsection{Symantec Backup Exec}

Symantec Backup Exec (formerly VERITAS Backup Exec) is a proprietary network
backup solution. It allows an administrator to backup multiple hosts to
a single tape unit or disk. The software has a long history starting with
Maynard Electronics\cite{Symantec-history}, and has recently added support for
continuous backup\cite{Symantec-about}.

Though it is primarily a Windows-based application, agents are available for
Linux/Unix, Solaris, Mac and Novel NetWare. A ``media agent'' is allow
available for Linux, allowing Linux servers to be used as a storage medium for
the archived files. The central management platform, however, runs only on
Microsoft Windows\cite{Symantec-about}.

A unique and proprietary storage format is used for the archives, which limits
recovery operations to proprietary Symantec tools. This also means that if the
backup server itself were to fail, the data would not be recoverable unless
a copy of the catalog were available.

Each Symantec agent must be separately licensed, which buys support from
Symantec for the software\cite{Symantec-about}.

\subsubsection{Bacula}

Bacula is an open-source network backup solution, based around a modular
architecture comprising the \emph{Director}, the \emph{Storage Server}, the
\emph{File Server}, the \emph{Monitor} and the
\emph{Console}\cite{Bacula-about}.

Similarly to Symantec, the project uses a unique storage format for archived
files that is not compatible with tools such as \emph{tar} or \emph{dump}
(though the Bacula format is open-source, and well documented). This means that
the storage server is not tied to a particular operating system, though backups
cannot be restored without the use of specially written Bacula restoration
tools. This is not considered a limitation by the developers, due to the
extensibility and superior capabilities of the format when compared with the
aforementioned standard formats\cite{Bacula-about}. Recovery, however, is made
more difficult if the Bacula software---or the software running the Bacula
server---were to fail, because the catalog is required to make sense of the
archive. Data can be restored without the catalog, but it requires a separate
``bootstrapping'' process, which adds to the complexity of the restoration.

The Bacula client agent (or \emph{File Server}) is available for multiple platforms,
including Linux, Solaris, Windows and Mac\cite{Bacula-features}.

Bacula lends itself to archiving files to a tape device in the same way as
Amanda.  For this reason, it does not support a version-centric backup model or
instant recovery. A web interface is, however, available---though it is
a separate project requiring independent installation and configuration---and
does not facilitate instant access to archived files through the
browser\cite{Bacula-about}.

\subsection{Cloud-based Backup Solutions}

\subsubsection{Mozy}

Mozy is an online backup service that allows users to backup their files to
``the cloud'' (meaning the Mozy servers on the internet)\cite{Mozy}. This is
a shift in paradigm from the traditional backup systems, which backup files to
a host on the local network, controlled by the user/administrator. 

Mozy provides client agent versions for both Windows and Mac, which support
continuous, scheduled or manual backups. A Linux client, however, is not
available\cite{Mozy}.

Once a file has been backed up, all changes in that file are reflected in the
archives. Users can restore previous versions of a file up to 30 days
old\cite{Mozy}.

Originally, Mozy was only available for home users. At present, however, Mozy
offers a ``pro'' versions for businesses also---which offers more features and
an administrative console\cite{Mozy-crunchbase}.

The pricing model for Mozy's products was originally based on a charge per
Gigabyte of data stored, on a monthly subscription
basis\cite{Carbonite-report}. This has now changed to a fixed monthly
subscription for unlimited data storage. A free version is also available, with
2GB of storage from up to 2 computers\cite{Mozy}.

Mozy offers a web interface for users to access their data, and download the
archived files for restoration. This can be done from any computer with access
to the internet\cite{Mozy}.

The shift to cloud-based storage, however, means that the transfer speed is limited to the upload speed
of the user's internet connection. This limit might not be
a problem for home users, but for businesses the delay between a file changing
and that change being archived could prove to be very expensive. Also, the
initial backup process is likely take a \emph{very} long time.

As a conservative example, uploading 500GB of data continuously, over
a connection with a 2Mbps upload speed would take over 7 months. This estimate
does not take into account other traffic using the connection, or any drop in
speed whatsoever. It is obvious that this initial transfer time could cause
major issues for an administrator with even a modest amount of data.

\subsubsection{Carbonite}

Carbonite is an online backup service similar to Mozy. The user's files are
stored on servers in Carbonite's datacenter, and the client agent support
Windows and Mac operating systems. Again, a Linux client agent is not
offered\cite{Carbonite}.

Carbonite began by offering a fixed, yearly subscription model for unlimited
storage capacity. This led to Mozy changing their pricing model to reflect
Carbonite's---leading to a very similar and competitive
service\cite{Carbonite-report}. A 30-day version history is also available for
all archived files\cite{Carbonite}.

A web interface is also offered, allowing users to view and restore their files
to any computer with access to the internet\cite{Carbonite}.

Because Carbonite uses the same storage model as Mozy, the backup speed is
again limited by the upload speed of the user's internet connection. Carbonite,
however, imposes extra limitations on the speed that a client can upload
data\cite{Carbonite-limits}. Continuing with the previous example, uploading
500GB of data continuously over a connection with a 2Mbps upload speed would
take approximately 10 months with Carbonite.

\subsection{File Synchronisation Services}

\subsubsection{Dropbox}

Dropbox is primarily a file synchronisation tool. It allows users to keep
a single directory tree synchronised between multiple computers (on different
platforms). The files are synchronised to the Dropbox servers, and then to all
other computers running the Dropbox client that are linked to user's
account\cite{Dropbox}.  The terms of service, however, do not disallow the use
of Dropbox for backing up data\cite{Dropbox-terms}.

The client and server code is proprietary, but the service is free for up to
2GB of storage. A ``freemium'' subscription model is used whereby more storage
can be purchased for a subscription charge\cite{Dropbox}.

The files that have been synchronised to Dropbox can be accessed via a web
interface which allows users to view the directory tree of their Dropbox folder
and view/restore archived versions of the files therein. A 30-day revision
history is allowed on the free plan, with unlimited versions available on the
paid subscription plan\cite{Dropbox}.

Dropbox suffers from the same limitation as Mozy and Carbonite, in that files
can only be transferred at the speed of the user's upload rate.

\subsubsection{Ubuntu One}

Ubuntu One is essentially a clone of Dropbox, solely for Ubuntu users---though
a Windows client is currently in development---with a paid mobile service for
Android and iPhone devices\cite{UbuntuOne}. It is created and supported by
Canonical, the company sponsoring the development of the popular Linux
operating system Ubuntu\cite{UbuntuOne}. Again, the terms of service do not
disallow using Ubuntu One as a backup solution\cite{UbuntuOne-terms}.

Although Ubuntu itself is open-source, Ubuntu One is a proprietary project. The
``agent'' that is installed on the client is open, but the server-side code is
not\cite{UbuntuOne-servers}. This means that users cannot run their own Ubuntu
One servers, and are forced to use the service provided by Canonical. Similarly
to Dropbox, more storage space can be purchased for a subscription
fee\cite{UbuntuOne}.

Ubuntu One also integrates with the local ``CouchDB'' instance on the Ubuntu
operating system, which allows it to synchronise data such as notes taken with
Tomboy, and contact data within Evolution\cite{UbuntuOne-couchdb}. Also,
a music store is available which, coupled with the paid mobile client, allows
users to stream music to their hand-held device over the
internet\cite{UbuntuOne}. This, however, is a feature that is unrelated to the
aims of this project.

As with Dropbox, a web interface is available for browsing the synchronised
files and folders and restoring prior versions of said files. The Ubuntu One
interface also summarises the recent activity with regards to the integrated
applications (Tomboy and Evolution)\cite{UbuntuOne}.

Finally, Ubuntu one is also limited by the upload speed of the user's internet
connection, in the same was as the previous systems.

\subsection{Other}

\subsubsection{Version Control}

Aside from the aforementioned systems, another category is related to the
development of this project. Collaborative version control services allow users
to host repositories on their servers, and provide an interface to browse the
code within those repositories. Social networking aspects are also featured in
such services, with users being able to ``follow'' the activity of others, or
interesting projects hosted on the website.

Examples of this type of service include GitHub\cite{Github} and
BitBucket\cite{Bitbucket}. While both are closed-source systems, the
``clients'', which are effectively the version control systems that are used to
contain the source code---git and mercurial respectively---are open source
projects themselves.

Though these services do not offer the ability to backup files in the same way
as the other systems mentioned above, they do require an intensely
version-centric interface, to allow users to browse the history of their source
code (and indeed the history of each file individually). This is effectively
the same interface that would be required for a version-centric backup system
such as this.

\subsubsection{Vembu StoreGrid}

Vembu StoreGrid is somewhat of a hybrid between the traditional backup systems
and the cloud-based backup solutions. It can be configured to backup to a local
server, and then replicate to an off-site location (including cloud services
such as Amazon S3). The payment model is based around a license (called an
MCAL), which costs a fixed amount and lasts for a finite time (i.e. a month).
MCALs are consumed by backup clients at a rate which is determined by the
functions performed by that particular client. For example, a Microsoft
Exchange server would consume 4 MCALs per month.

StoreGrid can backup files on a schedule, or in a continuous manner. The client
application is available for Windows, Mac, Linux, Solaris and FreeBSD.

Finally, StoreGrid allows service providers to sell a branded backup service to
their customers, charging for storage space on their own servers.

\subsection{Feature Comparison}

\begin{table}[H]
    \centering
    \begin{tabular}{ | l | p{1.5cm} | p{1.5cm} | p{1.5cm} | p{1.5cm} | }
        \hline
        & Open source   & Version control   & Web interface & Local network
            \\ \hline

        Amanda Network Backup   & $\times$  & $\cdot$   & $\cdot$   & $\times$
            \\ \hline

        Symantec Backup Exec    & $\cdot$   & $\times$  & $\cdot$   & $\times$
            \\ \hline

        Bacula                  & $\times$  & $\cdot$   & $\times$  & $\times$
            \\ \hline

        Mozy                    & $\cdot$   & $\times$  & $\times$  & $\cdot$
            \\ \hline

        Carbonite               & $\cdot$   & $\times$  & $\times$  & $\cdot$
            \\ \hline

        Dropbox                 & $\cdot$   & $\times$  & $\times$  & $\cdot$
            \\ \hline

        Ubuntu One              & $\cdot$   & $\times$  & $\times$  & $\cdot$
            \\ \hline

        Vembu StoreGrid         & $\cdot$   & $\times$  & $\times$  & $\times$
            \\ \hline
    \end{tabular}
    \caption{A comparison of features between a number of related works}
    \label{tab:feature-comparison}
\end{table}

\section{Technical Research}

\subsection{inotify}

In order to provide continuous protection for a filesystem, a backup program
must somehow know when a file has changed. With a small set of files, this can
be achieved by polling the filesystem at intervals. Obviously, this solution is
not viable for larger (and indeed more realistic) file sets.

Both Ubuntu One and the Linux version of Dropbox use a kernel subsystem called
\emph{inotify}, which allows applications to listen for filesystem
notifications, triggered when files are changed.

\emph{inotify} was released in the Linux kernel version 2.6.13 and was
developed as a replacement for \emph{dnotify}, which had many undesirable
flaws, and was widely considered difficult to use\cite{love2005}.

\putbib[research]

\end{bibunit}
