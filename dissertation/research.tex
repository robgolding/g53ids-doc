\section{Research}

\begin{bibunit}[plain]

\subsection{Related Works}

Related works fit into one of three main categories: Traditional backup
systems, on-line (cloud based) backup solutions, and file synchronisation
services.

\subsubsection{Traditional Backup Systems}

\paragraph{Amanda Network Backup}

Amanda (Advanced Maryland Automatic Network Disk Archiver) is an open-source,
multi-platform network backup solution\cite{AMANDA-about}. The software is
released under the Amanda Copyright and License (a BSD-style license) and is
supported by Zmanda\cite{AMANDA-license}.

Amanda uses native Unix tools to archive files (such as tar or dump) to disk or
tape on a central backup server. The client agent runs on almost all Linux or
Unix-based systems, while the server software only support Linux. Due the to
fact that it only uses native tools, no special drivers are required for
interfacing with devices such as tape libraries. Therefore, if the operating
system is able to communicate with the device (via commands such as
\textbf{mt}) then Amanda can use it.\cite{AMANDA-about}

The software lends itself heavily to the use of tape devices for backing up
data, and has some advanced features that facilitate this use. As such, it does
not support a version-centric backup model or
instant-recovery.\cite{AMANDA-about}

An ``enterprise'' version is also available---supported by Zmanda---which
offers a GUI management tool and enterprise-class support (SLA).  This version
is based on an annual subscription model\cite{AMANDA-ent}.

\paragraph{Symantec Backup Exec}

Symantec Backup Exec (formerly VERITAS Backup Exec) is a proprietary network
backup solution. It allows an administrator to backup multiple hosts to
a single tape unit or disk. The software has a long history starting with
Maynard Electronics, and has recently added support for continuous backup.

Though it is primarily a Windows-based application, agents are available for
Linux/Unix, Solaris, Mac and Novel NetWare. A ``media agent'' is allow
available for Linux, allowing Linux servers to be used as a storage medium for
the archived files. The central management platform, however, runs only on
Microsoft Windows.

Each Symantec agent must be separately licensed, which buys support from
Symantec for the software.

\paragraph{Bacula}

Bacula is an open-source network backup solution, based around a modular
architecture comprising the \emph{Director}, the \emph{Storage Server}, the
\emph{File Server}, the \emph{Monitor} and the \emph{Console}.

Unlike Amanda, the project uses a unique storage format for archived files that
is not compatible with tools such as \emph{tar} or \emph{dump}. This means that
the storage server is not tied to a particular operating system, though backups
cannot be restored without the use of Bacula restoration tools. This is not
considered a limitation by the developers, due to the extensibility and
superior capabilities of the format when compared with the aforementioned
standard formats.

The Bacula client agent (or \emph{File Server}) is available for multiple platforms,
including Linux, Solaris, Windows and Mac.

Similarly to Amanda, Bacula lends itself to archiving files to a tape device.
For this reason, it does not support a version-centric backup model or instant
recovery. A web interface is, however, available---though it is a separate
project that requires independent installation and configuration and does not
facilitate instant access to archived files through the browser.

\subsubsection{Cloud-based Backup Solutions}

\paragraph{Mozy}

Mozy is an online backup service that allows users to backup their files to
``the cloud'' (meaning the Mozy servers on the internet). This is a shift in
paradigm from the traditional backup systems, which backup files to a host on
the local network, controlled by the user/administrator.

Mozy provides client agent versions for both Windows and Mac, which support
continuous, scheduled or manual backups. A Linux client is not available.

Once a file has been backed up, all changes in that file are reflected in the
archives. Users can restore past versions of a file up to 30 days past.

Originally, Mozy was only available for home users. At present, however, Mozy
offers a ``pro'' versions for businesses also---which offers more features and
an administrative console.

The pricing model for Mozy's products was originally based on a charge per
Gigabyte of data stored, on a monthly subscription basis. This has now changed
to a fixed monthly subscription for unlimited data storage. A free version is
also available, with 2GB of storage from up to 2 computers.

Mozy offers a web interface for users to access their data, and download the
archived files for restoration. This can be done from any computer with access
to the internet.

\paragraph{Carbonite}

Carbonite is an online backup service similar to Mozy. The user's files are
stored on servers in Carbonite's datacenter(s), and the client agent support
Windows and Mac operating systems. Again, a Linux client agent is not offered.

Carbonite began by offering a fixed, yearly subscription model for unlimited
storage capacity. This led to Mozy changing their pricing model to reflect
Carbonite's---leading to a very similar and competitive service. A 30-day
version history is also available for all archived files.

A web interface is also offered, allowing users to view and restore their files
to any computer with access to the internet.

\subsubsection{File Synchronisation Services}

\paragraph{Dropbox}

Dropbox is primarily a file synchronisation tool. It allows users to keep
a single directory tree synchronised between multiple computers (on different
platforms). The files are synchronised to the Dropbox servers, and then to all
other computers running the Dropbox client that are linked to user's account.
The terms of service, however, do not disallow the use of Dropbox for backing
up data.

The client and server code is proprietary, but the service is free for up to
2GB of storage. A ``freemium'' subscription model is used whereby more storage
can be purchased for a subscription charge.

The files that have been synchronised to Dropbox can be accessed via a web
interface which allows users to view the directory tree of their Dropbox folder
and view/restore archived versions of the files therein. A 30-day revision
history is allowed on the free plan, with unlimited versions available on the
paid subscription plan.

\paragraph{Ubuntu One}

Ubuntu One is essentially a clone of Dropbox, solely for Ubuntu users---though
a Windows client is currently in development---with a paid mobile service for
Android and iPhone devices. It is created and supported by Canonical, the
company sponsoring the development of the popular Linux operating system
Ubuntu. Again, the terms of service do not disallow using Ubuntu One as
a backup solution.

Although Ubuntu itself is open-source, Ubuntu One is a proprietary project. The
``agent'' that is installed on the client is open, but the server-side code is
not. This means that users cannot run their own Ubuntu One servers, and are
forced to use the service provided by Canonical. Similarly to Dropbox, more storage space can
be purchased for a subscription fee.

Ubuntu One also integrates with the local ``CouchDB'' instance on the Ubuntu
operating system, which allows it to synchronise data such as notes taken with
Tomboy, and contact data within Evolution. Also, a music store is available
which, coupled with the paid mobile client, allows users to stream music to
their hand-held device over the internet.

As with Dropbox, a web interface is available for browsing the synchronised
files and folders and restoring prior versions of said files. The Ubuntu One
interface also summarises the recent activity with regards to the integrated
applications (Tomboy and Evolution).


\subsubsection{Feature Comparison}

\begin{table}[H]
    \centering
    \begin{tabular}{ | l | c | c | c | c | }
        \hline
        & Open source   & Version centric   & Web interface & Local network
            \\ \hline

        Amanda Network Backup   & $\times$  & $\cdot$   & $\cdot$   & $\times$
            \\ \hline

        Symantec Backup Exec    & $\cdot$   & $\times$  & $\cdot$   & $\times$
            \\ \hline

        Bacula                  & $\times$  & $\cdot$   & $\times$  & $\times$
            \\ \hline

        Mozy                    & $\cdot$   & $\times$  & $\times$  & $\cdot$
            \\ \hline

        Carbonite               & $\cdot$   & $\times$  & $\times$  & $\cdot$
            \\ \hline

        Dropbox                 & $\cdot$   & $\times$  & $\times$  & $\cdot$
            \\ \hline

        Ubuntu One              & $\cdot$   & $\times$  & $\times$  & $\cdot$
            \\ \hline
    \end{tabular}
    \caption{A comparison of features between a number of related works}
    \label{tab:feature-comparison}
\end{table}

\putbib[research]

\end{bibunit}
