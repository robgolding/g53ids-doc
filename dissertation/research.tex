\section{Research}

\subsection{Related Works}

Related works fit into one of three main categories: Traditional backup
systems, on-line (cloud based) backup solutions, and file synchronisation
services.

\subsubsection{Traditional Backup Systems}

\paragraph{Amanda Network Backup}

Amanda (Advanced Maryland Automatic Network Disk Archiver) is an open-source,
multi-platform network backup solution. It uses native Unix tools to archive
files (such as tar, dump etc.) to disk or tape on a central backup server.

The Amanda client agent runs on almost all Linux or Unix-based systems, and
there is also a native Windows client available. An ``enterprise'' version is
also available---supported by a company called Zmanda---based on an annual
subscription model.

\paragraph{Symantec Backup Exec}

Symantec Backup Exec (formerly VERITAS Backup Exec) is a more ``traditional''
network backup solution. It allows an administrator to backup multiple hosts to
a single tape unit or disk. The software has a long, proprietary history
starting with Maynard Electronics, and has recently added support for
continuous backup.

\paragraph{Bacula}

Bacula is an open-source network backup solution, based around a modular
architecture comprising the \emph{Director}, the \emph{Storage Server}, the
\emph{File Server}, the \emph{Monitor} and the \emph{Console}. The project uses
a unique storage format, so backups cannot be restored without specially
written tools.

\subsubsection{Cloud-based Backup Solutions}

\subsubsection{File Synchronisation Services}

\paragraph{Dropbox}

Dropbox is primarily a file synchronisation tool. It allows users to keep
a single directory tree synchronised between multiple computers (on different
platforms). The files are synchronised to Dropbox's servers, and then to all
other computers running the Dropbox client that are linked to user's account.

The client and server code is proprietary, but the service is free for up to
2GB of storage. A ``freemium'' subscription model is used whereby more storage
can be purchased for a subscription charge.

The files that have been synchronised to Dropbox can be accessed via a web
interface which allows users to view the directory tree of their Dropbox folder
and view/restore archived versions of the files therein. A 30-day revision
history is allowed on the free plan, with unlimited versions available on the
paid subscription plan.

\paragraph{Ubuntu One}

Ubuntu One is essentially a clone of Dropbox, solely for Ubuntu users---though
a Windows client is currently in development---with a paid mobile service for
Android and iPhone devices. It is created and supported by Canonical, the
company sponsoring the development of the popular Linux operating system
Ubuntu.

Although Ubuntu itself is open-source, Ubuntu One is a proprietary project. The
``agent'' that is installed on the client is open, but the server-side code is
not. This means that users cannot run their own Ubuntu One servers, and are
forced to use Canonical's service. Similarly to Dropbox, more storage space can
be purchased for a subscription fee.

Ubuntu One also integrates with the local CouchDB instance on the Ubuntu
operating system, which allows it to synchronise data such as notes taken with
Tomboy, and contact data within Evolution. Also, a music store is available
which, coupled with the paid mobile client, allows users to stream music to
their hand-held device over the internet.

As with Dropbox, a web interface is available for browsing the synchronised
files and folders and restoring prior versions of said files. The Ubuntu One
interface also summarises the recent activity with regards to the integrated
applications (Tomboy and Evolution).


\subsubsection{Feature Comparison}

\begin{table}[H]
    \centering
    \begin{tabular}{ | l | c | c | c | c | }
        \hline
        & Open source   & Version centric   & Web interface & Local network
            \\ \hline

        Amanda Network Backup   & $\times$  & $\cdot$   & $\cdot$   & $\times$
            \\ \hline

        Symantec Backup Exec    & $\cdot$   & $\times$  & $\cdot$   & $\times$
            \\ \hline

        Bacula                  & $\times$  & $\cdot$   & $\times$  & $\times$
            \\ \hline

        Dropbox                 & $\cdot$   & $\times$  & $\times$  & $\cdot$
            \\ \hline

        Ubuntu One              & $\cdot$   & $\times$  & $\times$  & $\cdot$
            \\ \hline
    \end{tabular}
    \caption{A comparison of features between a number of related works}
    \label{tab:feature-comparison}
\end{table}
