\section{Motivation}

Tradition Version Control Systems (VCS) can be used as a method for backing up
a collection of files, and storing a copy of each and every version as the
files change over time. In practice, however, this is very impractical. Every
time a file changes, the user must ``commit'' the working directory to the
revision control system, which then calculates the changes since the last
commit. Most (if not all) systems use a text-based diffing process, as they are
designed for tracking changes in \emph{source code}. This means that they are
inefficient when used with binary files, such as those that are typically found
in an organisation's file server tree.

This project aims to combine the features of traditional backup solutions with
those of Version Control Systems. The desired result is therefore a system
which keeps a historical archive of every version of every file in a given
directory tree on a remote system. This is something that, combined with an
intuitive interface, is not available in any open-source offerings.

The core feature-set of the system includes:

\begin{itemize}
    \item Version-based file recovery;
    \item Web interface for central management;
    \item Detailed reports regarding backed up data;
    \item Ability to recover files with standard Unix tools (tar, cp, etc.);
    \item Instant file recovery through web interface.
\end{itemize}
