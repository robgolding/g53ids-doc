\chapter{Implementation}

\section{Technologies}

\subsection{Django}

The core system implementation is based on the Django framework which, though
primarily designed for creating web applications, can be used as the basis for
any type of application.

\begin{quote}
    \emph{Django is a high-level Python Web framework that encourages rapid
    development and clean, pragmatic design.}\footnote{www.djangoproject.com}
\end{quote}

Django is based around a Model-View-Controller (MVC) architecture, separating
the business logic and data functions from the user interface.

In this project, Django is used to provide access to the database (the
\emph{models}) and for the web interface itself.

\subsubsection{Models}

The models define the persistent data that will be available to the
application, and how that data should be accessed. In Django, this consists of
a series of Python classes, one for each data ``entity'' (see Figure
\ref{fig:erd}).

\begin{singlespacing}
\begin{lstlisting}[caption=The `Event' model, label=lst:event-model]
    class Event(models.Model):
        item = models.ForeignKey(Item, db_index=True)
        occurred_at = models.DateTimeField(auto_now_add=True,
                                           db_index=True)
        type = models.CharField(max_length=20,
                                choices=EVENT_TYPE_CHOICES,
                                db_index=True)

        def __unicode__(self):
            return '%s %s' % (self.get_type_display(),
                              self.item.name)

        class Meta:
            ordering = ['-occurred_at']
            get_latest_by = 'occurred_at'
\end{lstlisting}
\end{singlespacing}

Listing \ref{lst:event-model} shows the definition of the \verb!Event! model,
which has only three attributes: \verb!item! (a foreign key to the \verb!Item!
model), \verb!occured_at! (the date/time at which the event occurred) and
\verb!type! (the type of the event).

The model definition also specifies how an instance of the model should be
displayed if it is printed on screen (the \verb!__unicode__()! method) and how
queries should be ordered (the \verb!Meta! class, defining \verb!ordering!
and \verb!get_latest_by!).

Django provides access to this data through its Object-Relational Mapper (ORM).
This provides a high-level API for querying the database, saving time and
reducing mistakes in the application code. Listing \ref{lst:orm-example} shows
an example of how one might query the database for all \verb!Event!s which
occurred on the \verb!Item! with ID 4, of type ``updated''.

\begin{singlespacing}
\begin{lstlisting}[caption=Querying for all ``updated'' Events on Item 4,
    label=lst:orm-example]
    Item.objects.filter(item__id=4, type='updated')
\end{lstlisting}
\end{singlespacing}

\subsubsection{Project Structure}

A Django project is separated into multiple ``applications'', each defining its
own models, views, and templates. The idea here is that applications should be
loosely coupled, allowing developers to simply ``plug in'' the desired
functionality, by enabling the relevant third-party applications.

This project is separated into three applications: \verb!core!, \verb!clients!,
and \verb!catalog!.

\subsection{Twisted}

\subsection{pyinotify}

\section{Server}

\subsection{Daemon}

\subsection{Storage}

\section{Client}

\subsection{Queues}

\section{Web Interface}
