\chapter{Implementation}

This project is implemented in Python, making use of the Django web framework
and the Twisted networking engine.

\section{Python}

\begin{figure}[h]
    \begin{verbatim}
    Beautiful is better than ugly.
    Explicit is better than implicit.
    Simple is better than complex.
    Complex is better than complicated.
    Flat is better than nested.
    Sparse is better than dense.
    Readability counts.
    Special cases aren't special enough to break the rules.
    Although practicality beats purity.
    Errors should never pass silently.
    Unless explicitly silenced.
    In the face of ambiguity, refuse the temptation to guess.
    There should be one-- and preferably only one --obvious way to do it.
    Although that way may not be obvious at first unless you're Dutch.
    Now is better than never.
    Although never is often better than *right* now.
    If the implementation is hard to explain, it's a bad idea.
    If the implementation is easy to explain, it may be a good idea.
    Namespaces are one honking great idea -- let's do more of those!
    \end{verbatim}
    \caption{The Zen of Python, by Tim Peters}
    \label{fig:zen-of-python}
\end{figure}

The backup system is implemented in Python, which is a fast, powerful and
dynamic programming language, suitable for a large variety of application
domains.

Python has a large and feature-rich standard library, and also has many
frameworks and third-party libraries available. Its clean, concise and readable
syntax also make it a pleasant programming language to work with.

Python was created by Guido van Rossum in the early 1990s. To ensure a clear,
easy-to-use language, van Rossum only used ideas that had proven their worth
over time in other programming languages. In particular, he wanted a language
that would be easy to extend for use with other languages.
\cite{lindstrom2005}.

Traditionally, systems of this nature would most likely be implemented in
a more ``classic'' language such as C or C++. Python offers a modern
alternative to these---not only in syntax but also in mindset.

Python libraries can be written in C, providing low-level access to
the internal Python bindings. This makes Python particularly suitable for use
with this project, as proof-of-concept code can be written in Python for
ease---and then rewritten in C later if performance is found to be an issue.

\section{Django}

The system implementation is based on top of the Django framework which, though
primarily designed for creating web applications, can be used as the basis for
any type of application.

\begin{quote}
    \emph{Django is a high-level Python Web framework that encourages rapid
    development and clean, pragmatic design.}\footnote{www.djangoproject.com}
\end{quote}

Django is based around a Model-View-Controller (MVC) architecture, separating
the business logic and data functions from the user interface.

In this project, Django is used to provide access to the database (the
\emph{models}) and for the web interface itself.

\subsection{Models}

The models define the persistent data that will be available to the
application, and how that data should be accessed. In Django, this consists of
a series of Python classes, one for each data ``entity'' (see Figure
\ref{fig:erd}).

\begin{singlespacing}
\begin{lstlisting}[caption=The `Event' model, label=lst:event-model]
    class Event(models.Model):
        item = models.ForeignKey(Item, db_index=True)
        occurred_at = models.DateTimeField(auto_now_add=True,
                                           db_index=True)
        type = models.CharField(max_length=20,
                                choices=EVENT_TYPE_CHOICES,
                                db_index=True)

        def __unicode__(self):
            return '%s %s' % (self.get_type_display(),
                              self.item.name)

        class Meta:
            ordering = ['-occurred_at']
            get_latest_by = 'occurred_at'
\end{lstlisting}
\end{singlespacing}

Listing \ref{lst:event-model} shows the definition of the \verb!Event! model,
which has only three attributes: \verb!item! (a foreign key to the \verb!Item!
model), \verb!occured_at! (the date/time at which the event occurred) and
\verb!type! (the type of the event).

The model definition also specifies how an instance of the model should be
displayed if it is printed on screen (the \verb!__unicode__()! method) and how
queries should be ordered (the \verb!Meta! class, defining \verb!ordering!
and \verb!get_latest_by!).

Django provides access to this data through its Object-Relational Mapper (ORM).
This provides a high-level API for querying the database, saving time and
reducing mistakes in the application code. Listing \ref{lst:orm-example} shows
an example of how one might query the database for all \verb!Event!s which
occurred on the \verb!Item! with ID 4, of type ``updated''.

\begin{singlespacing}
\begin{lstlisting}[caption=Querying for all ``updated'' Events on Item 4,
    label=lst:orm-example]
    Event.objects.filter(item__id=4, type='updated')
\end{lstlisting}
\end{singlespacing}

\subsection{Views}

Django uses a slightly different terminology to other popular frameworks, in
that the ``controllers'' (the pieces of code which link together the models and
the user interface) are termed \emph{views}.

The view's purpose is to parse the incoming HTTP request, extract the relevant
data (if any) from the data models, and respond with an appropriate HTTP
response. This is usually constructed by rendering a template.

Fairly obviously, the views are only relevant to the web interface, and are not
used for the core backup system functionality.

\subsection{Templates}

A template (often called a ``view'' in other frameworks) is a file which
defines how the interface will be rendered to the end user. When dealing with
the web, this is usually an HTML file containing variables and simple logic.

Again, the templates are not relevant to the core system functionality---and
only relate to the web interface.

\subsection{Project Structure}

A Django project is separated into multiple ``applications'', each defining its
own models, views, and templates. The idea here is that applications should be
loosely coupled, allowing developers to simply ``plug in'' the desired
functionality by enabling the relevant third-party applications.

This project is separated into three applications: \verb!core!, \verb!clients!,
and \verb!catalog!.

\subsubsection{Core Application}

The core application exists to tie in the functionality provided by the other
two applications, and to provide functionality which does not necessarily
belong in either of them.

The only model defined in the core application is the \verb!GlobalExclusion!,
as it is not directly related to any other model (or the catalog or clients
functionality for that matter).

The core application contains the code for the \emph{dashboard} view, which
provides an overview of the system status and some basic statistics. Also, the
\emph{configuration} view is implemented here---as this is a core system
feature, and not related to the clients or catalog applications.

\subsubsection{Clients Application}

The clients application provides functionality related to the clients
themselves, such as the \verb!Client! model (along with the related
\verb!Status!, \verb!FilePath! and \verb!Exclusion! models) and the views which
allow the administrator to add clients, and edit or delete existing ones.

More detail as to the implementation of the web interface itself is given in
section \ref{sec:impl-web-interface}.

\subsubsection{Catalog Application}

The catalog application provides the main persistent data store for the backup
application (the \emph{catalog}) which records the files and versions that have
been archived from each client. At the core, this consists of the \verb!Item!
and \verb!Version! models. The \verb!Event! and \verb!RestoreJob! models are
also included in the catalog application. See Figure \ref{fig:erd} for the
detailed Entity-Relationship Diagram.

\section{Twisted}

Though the backup system is implemented on top of Django, the core
functionality is achieved using the Twisted framework.

Twisted is an open-source event-driven networking engine written in Python. It
allows programmers to write highly asynchronous programs without resorting to
multiple threads or processes, which add to the overall complexity of the
application \cite{kinder2005}.

Due to the asynchronous nature of the framework, programming with Twisted is
different to most other libraries. Central to the framework is the concept of
a \emph{deferred}, which represents an event that is yet to happen. This allows
method calls to return without actually completing their task, and for a given
piece of code to be executed once that task has ended.

At the heart of the Twisted framework is the \emph{reactor}---the main event
loop. This allows the entire application to be ``hooked'' into a central
location, from which all events and callbacks are generated or called.

\subsection{Perspective Broker}

The Twisted framework contains a library for networked inter-process
communication called the \emph{Perspective Broker}. Essentially, it is
a Twisted-specific protocol for Remote Procedure Call (RPC). This allows
programs at either end of the network connection to call methods (procedures)
on an object that exists within the remote process. \cite{lefkowitz2003}

The Perspective Broker is used for all inter-process communication within the
backup system, including the actual data transfer itself.

\subsection{Plugins}
\label{sec:twisted-plugins}

Twisted allows developers to create application ``plugins'', which are
networked programs that can be launched using the twisted daemon \emph{twistd}.
This removes the need for complex daemonization routines (the classic
``double-fork'') in the main application code, and eases the management of PID
and log files.

Each of the three system components (the server, client, and web interface) are
written as Twisted plugins. This means that they can be launched independently
of one-another, and maintain their own process table (PID, log file, etc.) An
example process invocation is shown in listing \ref{lst:twistd-example}.

\begin{singlespacing}
\begin{lstlisting}[caption=An example twistd invocation,
    label=lst:twistd-example]
    twistd \
        --pidfile /var/run/backtracd.pid \
        --logfile /var/log/backtracd.log \
        backtracd
\end{lstlisting}
\end{singlespacing}

\section{pyinotify}

\emph{pyinotify} is an open-source library for Python which provides
a high-level API for the inotify subsystem. It allows the programmer to add
watches to a directory in a recursive manner, and to automatically add watches
to newly created directories below the parent path.

Though Twisted contains a module for this exact same purpose, it suffers from
a bug which causes file system events to be fired multiple times in error. For
this reason pyinotify is used to monitor the file system in this project
instead, and is linked into the Twisted event loop with a custom hook.

\section{Server}

The server provides the sole point of contact for the client application,
including all data transfer operations. The system is designed such that all
\emph{business logic} (i.e. determining which files to backed up) is contained
within the server. This makes for simpler, faster, and more secure clients.

The server application is implemented primarily as a Perspective Broker, and
leverages an API which serves as an abstraction between the server itself and
the data models. This means that the code is extensible, in that the server
need not be modified if the schema changes.

\subsection{API}

The API that sits between the server and the Django models works by
encapsulating backup functionality into a \verb!Client! object, which can then
be \emph{accessed} or \emph{mutated} as necessary.

The methods that the \verb!Client! API provides are as follows:

\begin{itemize}
    \item \verb!connected()! - the client has connected to the server
    \item \verb!disconnected()! - the client has disconnected from the server
    \item \verb!get_hostname()! - get the hostname of the client
    \item \verb!get_key()! - get the secret authentication key of the client
    \item \verb!get_paths()! - get the paths which are to be protected on the
        client
    \item \verb!get_exclusions()! - get the exclusions which are defined for
        the client
    \item \verb!get_present_state(path)! - get the present state of the
        client's file system from \verb!path! downwards, according to the
        catalog (including each item, and it's meta-data)
    \item \verb!is_excluded(path)! - check whether \verb!path! matches any
        exclusions (global or local) on the client
    \item \verb!create_item(path, type)! - create an item of type \verb!type!
        at \verb!path! in the client's file system catalog
    \item \verb!update_item(path, mtime, size, version_id)! - update the item
        at \verb!path! with the given meta data, and associate the new version
        with \verb!version_id! (for retrieving the file from the storage
        subsystem)
    \item \verb!delete_item(path)! - delete the item at \verb!path! in the
        client's file system catalog (marks the item as deleted in the catalog,
        rather than actually deleting it!)
    \item \verb!compare_attrs(old_attrs, new_attrs)! - calculates the
        difference between \verb!old_attrs! and \verb!new_attrs! as an integer
        signifying the amount of difference ($0$ meaning equal, $n$ meaning $n$
        attributes differ)
    \item \verb!backup_required(path, attrs)! - check whether the item at
        \verb!path! with attributes \verb!attrs! is in need of a backup (uses
        \verb!compare_attrs! to compare the attributes with those stored in the
        catalog)
    \item \verb!get_pending_restore_jobs()! - get a list of pending restore
        jobs for the client, in the format
        \verb!(job_id, source_path, version_id, destination_path)!
    \item \verb!restore_begin(job_id)! - notify the server that the restore job
        with ID \verb!job_id! has begun
    \item \verb!restore_complete(job_id)! - notify the server that the restore
        job with ID \verb!job_id! has completed
\end{itemize}

\subsection{Daemon}

As mentioned, the server daemon provides the sole point of contact for the
client application, and is implemented as a Twisted \emph{Perspective Broker}.

The server is encapsulated within a \verb!Server! object, which provides a hook
into Twisted's \emph{service} architecture. This allows the plugin (see section
\ref{sec:twisted-plugins}) to start the server when the application is run, and
begin listening on the desired port.

The server also handles all authentication automatically, by ensuring that
an incoming client connection matches a record in the database with the correct
secret key.

Finally, the server periodically checks for pending restore jobs for each
connected client - executing them if necessary.

\subsection{Perspective Broker}

The server represents each connected client with a \verb!BackupClient! object.
This object maintains a reference to the previously mentioned \verb!Server!,
and the \verb!Client! API object which provides access to the catalog and
system internals.

The perspective broker provides the following remotely-accessible methods,
which can be called by the client:

\begin{itemize}
    \item \verb!get_paths()! - get the paths which are to be backed up by this
        client
    \item \verb!get_present_state(path)! - get the present state of the
        client's file system from \verb!path! downwards, according to the
        catalog (including the name of each item, but \textbf{no} meta data)
    \item \verb!check_index(path, cur_index)! - check the given index
        (\verb!cur_index!) against that which exists in the catalog, retrieving
        the names of the files which require archival
    \item \verb!check_file(path, attrs)! - check if the file at \verb!path!
        with the attributes in \verb!attrs! requires backing up
    \item \verb!create_item(path, type)! - notify the server that the item at
        \verb!path! with type \verb!type! has been created on the file system
    \item \verb!delete_item(path)! - notify the server that the item at
        \verb!path! has been deleted from the file system (or moved to another
        location)
    \item \verb!put_file(path, mtime, size)! - transfer the file at \verb!path!
        (with the given \verb!mtime! and \verb!size!) to the archive; returns
        a remotely-referenceable \verb!PageCollector! object (see section
        \ref{sec:data-transfer}).
\end{itemize}

\subsection{Data Transfer}
\label{sec:data-transfer}

To effect the actual transfer of data between the client and server for backing
up files (and vice-versa, in the case of restoring files), a custom \emph{File
Pager} is used. This is an object which which leverages the Perspective Broker
to ``page'' the file to the destination over the network.

Paging consists of reading the source file in chunks, and passing each chunk
over the network one-by-one.

On the sender, the \verb!TransferPager! reads from the source file in chunks
(of $2^{14} = 16,384$ bytes each), and places them into a \verb!PageCollector!,
which is a remote reference to an object that exists on the destination.

\subsection{Storage}
\label{sec:implementation-storage}

The storage subsystem acts as an abstraction layer between the backup server
and the file system. In essence, it allows multiple versions of many files to
be placed in ``buckets'', with each version being assigned a unique identifier.
These versions can be retrieved at a later time, given the three components
that identify it (bucket name, file name, version ID). The storage subsystem's
design is discussed in section \ref{sec:design-storage}.

The storage subsystem is encapsulated into a \verb!Storage! object, which is
initialised with the root of directory within which the subsystem should
operate. This allows multiple data stores to be maintained if required.

\subsubsection{Buckets}

Buckets are essentially top-level directories which separate files into logical
groups. In practice, each client's files are contained within a dedicated
bucket.

The directory that represents a bucket is given the same name as the bucket
itself. For this reason, the bucket name can only contain characters which are
valid on the file system.

\subsubsection{Containers}

At the next level down, containers are used to separate each file's versions
from others. Similarly, a container is simply a sub-folder of the bucket that
contains it.

As containers represent files themselves, a simple naming scheme such as that
used for buckets would be unsuitable. Crucially, the separators in the file's
name would clash with the server's file system---so a different approach is
required.

The name of a container is constructed by performing a one-way hash of the
file's name. For this, the SHA-1 hash function is used, as it has a low
probability of collisions \textbf{[REFERENCE]} and support for it is widely
available.

\subsubsection{Versions}

Within each container, every version of a file is stored. The files are named
according to the version's unique ID, with no file extension. This enforces the
uniqueness of the identifier, as two files with the same name cannot reside in
the same directory.

\subsubsection{put}

\subsubsection{get}

\section{Client}

\subsection{Queues}

\section{Web Interface}
\label{sec:impl-web-interface}
