\chapter{Specification}

This Software Requirements Specification (SRS) was produced with the backing of
\textbf{Servat Ltd.}, of Nottingham.

Servat is primarily an accounting system provider, installing and maintaining
software such as DataFile, Aqilla and SAP for their clients. They also resell
a remote backup service to their customers, based on the \emph{Vembu StoreGrid}
system. This puts the company in an ideal position to offer advice for this
project.

\section{Overview}

\emph{Backtrac Backup System} is an open-source backup system for small to
medium-sized installations, which focuses on version control for individual
files. The system is comprised of two main components: the \emph{server} (of
which there is one per installation), and the \emph{client} (of which there
may be many).

The clients are the computers which contain important data, to be
protected by the system. This data is transferred over the network, and stored
on the server.

The administrator can access the backups via a web interface, which allows
him or her to browse the files that have been archived for all protected
clients.

\section{System Context}

The system will have only minimal interaction with other systems, and all
aspects of the system itself will be self-contained. A dedicated database of
user authentication information will be maintained, meaning that external
factors do not affect the ability of the administrator(s) to access backups.

It is envisaged that the backup server will run on a dedicated hardware
machine, with plentiful storage (ideally in a RAID configuration to guard
against disk failures).

\section{User Requirements}

\subsection{Scenarios}

 - Some user scenarios would be nice. Include *far* too much detail. -

\section{System Architecture}

The system will be composed of three major parts, which operate independently
of one another:

\begin{itemize}
    \item The backup server daemon
    \item The backup client daemon
    \item The backup server web interface
\end{itemize}

\subsection{Server Daemon}

The backup server daemon is the program which will accept incoming connections
from backup clients, and store the archived files on the server's hard disk(s).
It will also keep an up-to-date catalog of all protected files, which can be
accessed by an administrator via the web interface.

\subsubsection{Meta Data}

The server daemon will allow clients to check whether a local file has been
modified, by comparing the file's meta-data to that stored in the catalog. This
meta-data may consists of any or all of the following attributes:

\begin{itemize}
    \item File size
    \item File modification time
    \item Inode number
    \item A hash or checksum of the file's contents
\end{itemize}

This information may then be used to determine whether the file needs to be
archived on the server, or if the current version of the archive is up-to-date.

\subsubsection{Current State}

To allow the client to compare the current filesystem's contents with that of
the most up-to-date backup, the server will also allow clients to retrieve the
state of the catalog at the present moment in time. This ``state'' need not
include the meta-data of the archived files, but only their path names---as the
previously mentioned check may be performed to determine whether a backup is
required given that a file exists. The main purpose of this operation is to
notify the server of any files which have been deleted on the client
filesystem, but still remain in the catalog. This may be as a result of the
client daemon being stopped for some period of time.

\section{System Requirements}

\section{System Models}

\section{System Evolution}

\section{Non Goals}

The system (at least this particular version) will \emph{not} provide the
following features:

\begin{itemize}
    \item Bare-metal disaster recovery for entire operating systems
    \item Cloud support for off-site data storage (e.g. Amazon S3)
\end{itemize}
