\documentclass[a4paper]{article}

\usepackage{url}
\usepackage{graphicx}
\usepackage{enumerate}
\usepackage{amsmath}
\usepackage{float}
\usepackage{longtable}
\usepackage{fullpage}

\title{
    \vspace{3cm}
    \huge{G53IDS Individual Dissertation Single Honours} \\[0.5cm]
    \LARGE{Project Proposal} \\[0.2cm]
}
\author{Robert J. Golding (rjg08u)} \date{\today}

\begin{document}
    \maketitle
    \vspace{10cm}
    \begin{center}
        \includegraphics{notts.jpg}
    \end{center}
    \newpage

    \section{Background \& Motivation}
    There are many backup systems available in both the proprietary and open
    source worlds. However, I have long felt that there are no open source
    systems that meet all the needs of the modern enterprise environment.
    Backup systems are often thought of as `set-and-forget', though this is far
    from the case in a typical IT department. Users are often requesting that
    accidentally deleted files be restored, and administrators require detailed
    reports to be sure that their data is safe at all times.

    This gap in the market first came to my attention when I was designing
    a network for my home in 2007, consisting of multiple Linux servers, each
    with files that needed backing up. I was unable to find an open source
    backup solution that incorporated a central web interface for management
    and reporting, that offered a simple setup procedure.

    Eventually, I implemented a custom backup library in \emph{Python}, which
    was in essence a simple wrapper around the \emph{rsync} program. This
    system works at a most basic level, but is wholly inadequate for its
    intended purpose. It consists of a simple script which calls \emph{rsync}
    to copy files to a selected location. As the administrator, I have no way
    of viewing the files that have been copied, save for manually inspecting
    log files.

    I would now like to take this project forward, by developing the system
    I was unable to find three years ago.

    \section{Aims \& Objectives}
    I propose to develop a backup system to address the inadequacies present in
    the currently available open source solutions:

    \begin{itemize}
        \item Web interface providing central management and reporting
        \item Detailed reports and historical data for every backup
        \item Instant file recovery offered through web interface, for
            accidentally deleted files etc.
        \item Ability to recover files without any special software or tools
    \end{itemize}

    \section{Project Plan}

\end{document}
