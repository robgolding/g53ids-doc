\documentclass[a4paper]{article}

\usepackage{url}
\usepackage{graphicx}
\usepackage{enumerate}
\usepackage{amsmath}
\usepackage{float}
\usepackage{longtable}
\usepackage{fullpage}

\title{
    \vspace{3cm}
    \huge{G53IDS Individual Dissertation Single Honours} \\[0.5cm]
    \LARGE{Project Proposal} \\[0.2cm]
}
\author{Robert J. Golding (rjg08u)} \date{\today}

\begin{document}
    \maketitle
    \vspace{10cm}
    \begin{center}
        \includegraphics{notts.jpg}
    \end{center}
    \newpage

    \section{Background \& Motivation}
    There are many backup systems available in both the proprietary and open
    source worlds. However, I have long felt that there are no open source
    systems that meet all the needs of the modern enterprise environment.
    Backup systems are often thought of as `set-and-forget', though this is far
    from the case in a typical IT department. Users are often requesting that
    accidentally deleted files be restored, and administrators require detailed
    reports to be sure that their data is safe at all times.

    This gap in the market first came to my attention when I was designing
    a network for my home in 2007, consisting of multiple Linux servers, each
    with files that needed backing up. I was unable to find an open source
    backup solution that incorporated a central web interface for management
    and reporting, that offered a simple setup procedure.

    Eventually, I implemented a custom backup library in \emph{Python}, which
    was in essence a simple wrapper around the \emph{rsync} program. This
    system works at a most basic level, but is wholly inadequate for its
    intended purpose. It consists of a simple script which calls \emph{rsync}
    to copy files to a selected location. As the administrator, I have no way
    of viewing the files that have been copied, save for manually inspecting
    log files.

    I would now like to take this project forward, by developing the system
    I was unable to find three years ago.

    \section{Aims \& Objectives}
    I propose to develop a network backup system to address the inadequacies
    present in the currently available open source solutions:

    \begin{itemize}
        \item Web interface providing central management and reporting
        \item Detailed reports and historical data for every backup
        \item Instant file recovery offered through web interface, for
            accidentally deleted files etc.
        \item Ability to recover data without any special software or tools
        \item Version control for individual files, allowing administrator to
            restore a specific version of a lost file
    \end{itemize}

    For me, the thing that is missing from the current offerings is
    \emph{data}. My aim is to ensure that I develop a system that offers users
    a large amount of data---and to design an interface that facilitates the
    viewing of this data in a quick and easy way. This will inevitably require
    a database backend, which may require careful management to ensure that
    performance does not degrade with the amount of data that is stored.

    As an example, a large amount of information could be summarized on
    a `dashboard' page, with graphs and/or charts displaying data relating to
    different aspects of the system (such as total bytes protected).

    Perhaps the most important aspect of my proposal however, is to include
    a form of version control in the backup process. I have recently been
    heavily involved in web applications development, and have therefore been
    introduced to VCS systems such as Subversion and Git. It became obvious to
    me that to combining these two technologies would result in a very useful
    system for data protection.

    Ideally, I envisage a system whereby the administrator searches for a file using the
    built-in web interface, and is then able to view all known versions of that
    file. He/she will then be able to restore the desired version. I feel this
    is an important feature that is missing from the open-source backup tools
    at present.

    \section{Project Plan}

\end{document}
