\documentclass[a4paper]{article}

\usepackage{url}
\usepackage{graphicx}
\usepackage{enumerate}
\usepackage{amsmath}
\usepackage{float}
\usepackage{longtable}
\usepackage{fullpage}
\usepackage{fontspec}

\setromanfont{Gentium}

\title{
    \vspace{3cm}
    \huge{G53IDS Individual Dissertation Single Honours} \\[0.5cm]
    \LARGE{Project Proposal} \\[0.2cm]
}
\author{Robert J. Golding (rjg08u)} \date{\today}

\begin{document}
    \maketitle
    \vspace{10cm}
    \begin{center}
        \includegraphics{notts.jpg}
    \end{center}
    \newpage

    \section{Background \& Motivation}
    There are many backup systems available, in both the proprietary and
    open-source worlds. However, I have long felt that no one system meets all
    the needs of the modern enterprise environment, with a focus on central
    management, control and detailed reporting.

    This gap in the market first came to my attention when I was designing
    a network for my home, which consisted of multiple Linux servers, each with
    files that needed backing up. The leading open-source contender seemed, at
    the time, to be Amanda Open-Source Backup. Amanda, however, did not include
    a web interface, and as a result was difficult to configure.

    I have since made multiple attempts at creating a backup system that would
    fulfill these requirements, though I have not yet been successful.

    \section{Aims \& Objectives}
    I propose to build a system to backup files on a local network. The backup
    system will be centrally managed via a web application, and will store
    detailed reports about what was backed up after every run.

    \section{Project Plan}

\end{document}
